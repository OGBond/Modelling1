%% LyX 2.2.3 created this file.  For more info, see http://www.lyx.org/.
%% Do not edit unless you really know what you are doing.
\documentclass[english]{article}
\usepackage[T1]{fontenc}
\usepackage[latin9]{inputenc}
\usepackage{geometry}
\geometry{verbose,tmargin=2cm,bmargin=2cm,lmargin=2cm,rmargin=2cm}
\usepackage{amsmath}
\usepackage{graphicx}
\usepackage{babel}
\begin{document}

\title{\textbf{Modelling Challenge:} Fluids on membranes}

\author{Oliver Bond, Meredith Ellis, Nicolas Bouell\'e, Huining Yang}
\maketitle

\subsection*{Assumptions}
\begin{itemize}
\item The \textbf{pressure }due to the liquid is \emph{hydrostatic }(so
there is \textbf{no fluid flow})
\item The upward force is only due to the surface tension of fluid
\item All the fibres are \textbf{square in cross-section }and of the \emph{same
size}
\item There is only one layer of fibres and these are in 1 dimension
\item The membrane \textbf{does not bend}
\item The pressure is constant over the widthscale of the fibre
\item The flow is incompressible and irrotational
\end{itemize}

\subsection*{Physical law}

We consider \textbf{Newton's Second Law}, in other words, balancing
forces. 

At each point on the surface, pressure balance results in (at each
point on the interface)
\[
\rho gh=\kappa\gamma
\]

where $\rho$ is the density of water, $\kappa$ is the curvature
of the interface, $\gamma$ is a constant, and $h=h(x)$ is the vertical
distance of the meniuscus at point $x$ from the bottom edge of one
of the fibres. In particular (from page 13 of the \emph{Topics in
Fluids }notes), $\kappa$ is the curvature of the meniscus, given
by

\[
\kappa=\dfrac{h_{xx}}{\left(1+h_{x}^{2}\right)^{\frac{3}{2}}}.
\]

However, in the simplest possible case we can assume that this is
a constant, that is, that the meniuscus is a circular arc. So we can
use $\kappa=\dfrac{1}{R_{M}}$ where $R_{M}$ is the radius of this
circular arc. All that remains is to find $R_{M}$ in terms of the
other parameters, which can be done using trigonometry. 

\subsection*{Constitutive law}

We want to find the steady state of the membrane. To start with we
assume that the pressure is hydrostatic, i.e. $h=h(x)$.

\subsection*{Variables}
\begin{itemize}
\item $\alpha$ is the angle (in \emph{radians}) between each point of contact
of the meniscus and fibre, and the horizontal axis pointing towards
the middle of the two fibres. If these are different for each of the
fibres, we shall call these $\alpha_{1}$ and $\alpha_{2}$.
\item $R_{M}$ is the (signed) radius of curvature of the meniuscus (in
\emph{metres}) when it is in its steady state. If $R_{M}>0$ then
the meniscus is $\cup$-shaped; if $R_{M}<0$ then the meniscus is
$\cap$-shaped.
\end{itemize}

\subsection*{Parameters}
\begin{itemize}
\item $d_{w}$ is the distance between the fibres (in \emph{metres})
\item $R$ is the radius of each of the fibres (in \emph{metres}). If the
radii of the two fibres are different, we shall call these $r_{1}$
and $r_{2}$.
\item $\theta$ is the angle of contact (in \emph{radians}) between the
meniscus and the fibre (i.e. if there is a point where the meniscus
touches the fibre, this is the angle between the tangent to the circle
at that point, and the tangent of the meniscus at that point).
\item $\ell$ is the distance (in \emph{metres}) between the centres of
two fibres.
\end{itemize}

\subsection*{Constitutive law}

We want to find the steady state of the membrane. To start with we
assume that this is a circular arc, i.e. $R_{M}$.

\subsection*{Solution}

We can use trigonometry to derive the formula
\[
R_{M}=\dfrac{\dfrac{\ell}{2}-R\cos\alpha}{-\cos(\alpha-\theta)}.
\]

This specifies the function $R_{M}$ in terms of $\alpha$.

There is a value of $\alpha$, which we will call $\alpha_{{\rm crit}}$,
for which the curvature changes sign (so the meniscus changes from
being $\cap$-shaped to $\cup$-shaped). In between these cases, the
meniscus is completely flat, which happens as $\left|R_{M}\right|\to\infty$.
This happens when $\cos(\alpha-\theta)=0$, i.e. when $\alpha-\theta=\frac{\pi}{2}$

so

\[
\alpha_{{\rm crit}}=\dfrac{\pi}{2}+\theta.
\]

\end{document}
